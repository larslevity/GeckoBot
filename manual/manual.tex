\documentclass[
	fontsize=10pt
	paper=a4
]{scrartcl} 

\usepackage[utf8]{inputenc} 
\usepackage{lmodern}

\usepackage[rmargin=2cm, lmargin=2cm, tmargin=3cm, bmargin=3cm]{geometry}

\usepackage{array}
\usepackage{tikz}
\usepackage{listings}
\usepackage{xcolor}
\definecolor{zebg}{rgb}{1,1,.8} %elfenbeinfarbig
\usepackage{url}
\usepackage{booktabs}

\usepackage{ragged2e}
\newcolumntype{P}[1]{>{\RaggedRight\hspace{0pt}}p{#1}}
\newcolumntype{C}[1]{>{\centering\arraybackslash}p{#1}}

\lstset{language=Matlab, numbers=left, numberstyle=\tiny, basicstyle=\footnotesize,showstringspaces=false,%
 numberblanklines=false, frame=single,
 backgroundcolor=\color{zebg},xleftmargin=0cm,xrightmargin=2cm,
 breaklines=true,
 linewidth=1.11\linewidth}

\usepackage{hyperref}	% Einträge der Inhaltsangabe verlinken; als letztes aller Pakete laden!
\hypersetup{
  colorlinks   = false, %Colours links instead of ugly boxes
  urlcolor     = black, %Colour for external hyperlinks
  linkcolor    = black, %Colour of internal links
  citecolor   = black %Colour of citations
}
\renewcommand{\UrlBreaks}{\do\-\do\?\do\/\do\a\do\b\do\c\do\d\do\e\do\f\do\g\do\h\do\i\do\j\do\k\do\l\do\m\do\n\do\o\do\p\do\q\do\r\do\s\do\t\do\u\do\v\do\w\do\x\do\y\do\z\do\A\do\B\do\C\do\D\do\E\do\F\do\G\do\H\do\I\do\J\do\K\do\L\do\M\do\N\do\O\do\P\do\Q\do\R\do\S\do\T\do\U\do\V\do\W\do\X\do\Y\do\Z}

\setlength{\parindent}{0pt}

\author{Lars Schiller}
\title{GeckoBot Code Manual}

\begin{document}

\maketitle
\tableofcontents




\section{Setting Up the BBB}




\subsection{Install OS on BBB}

The developers of BBB embedded linux systems decided to change the device tree structure from \texttt{kernel} overlay (till version 8.7), to \texttt{uboot} overlay (9.1+). (Don't ask me to explain).
However, the PWM setup for all pins is only possible with \texttt{kernel} overlay (or at least I'm not able to configure it in version 9.1+).
Therefore you have to use the following image:

\texttt{bone-debian-8.7-iot-armhf-2017-03-19-4gb.img}
(Download: \url{http://beagleboard.org/latest-images})

To install it on a \texttt{8GB} Micro-SD Card follow the instructions:

\begin{itemize}
\item You can use Etcher (\url{https://etcher.io/}).
\end{itemize}

OR (on debian):

\begin{itemize}
\item Instructions from:~~~\url{http://derekmolloy.ie/write-a-new-image-to-the-beaglebone-black/}

and from:~~\url{https://learn.adafruit.com/beaglebone-black-installing-operating-systems?view=all#copying-the-image-to-a-microsd}

\item Decompress and write on SD card (need to be \texttt{su} and make sure the security locker of SD Adapter is in writing mode):
\begin{lstlisting}
$ xz -d bone-debian-**.img.xz
$ dd if=./bone-debian-**.img of=/dev/sdX
\end{lstlisting}

(Here, \texttt{sdX} is the mounted empty uSD Card. It can be found with multiple use of the command \texttt{mount} or \texttt{df}.)


\item Obsolete:
\begin{footnotesize}
\begin{itemize}

\item In order to turn these images into eMMC flasher images, edit the \texttt{/boot/uEnv.txt} file on the BBB and remove the \texttt{\#} on the line with 

\texttt{cmdline=init=/opt/scripts/tools/eMMC/init-eMMC-flasher-v3.sh}. 

Enabling this will cause booting the microSD card to flash the eMMC. Images are no longer provided here to avoid people accidentally overwriting their eMMC flash.

\item Insert the SD Card in the unpowered BBB, and power it by plugging in the USB or the 5VDC supply. Wait until all 4 LED have solid lights. This can take up to 45 minutes. 


\item Flash MicroSD 4 with: \texttt{Debian 8.7 2017-03-19 4GB SD IoT} from \url{http://beagleboard.org/latest-images} (MicroSD 3 is weird ...).

\item Insert MicroSD in (unpowered) BBB, press the USER Button, and apply power.

\item It will take 30-45 minutes to flash the image onto the on-board chip. Once it is done, the bank of 4 LEDs to the right of the Ethernet port will all turn off. You can then power down your BBB.

\end{itemize}

\end{footnotesize}

\end{itemize}



\subsection{Log in BBB for the first time}

Assuming you are called \texttt{bianca} and your PC is also called \texttt{bianca},
your BBB is called \texttt{beaglebone} and the default user on BBB is called \texttt{debian}, then the following sythax is correct.

\begin{itemize}
\item Connect your PC with a MicroUSB cable to the BBB.

\item Open a terminal and ssh into BBB as \texttt{debian} and then get superuser to configure the Board.
\begin{lstlisting}
bianca@bianca:~ ssh debian@192.168.7.2
temppwd
debian@beaglebone:~ su
root
root@beaglebone:~#
\end{lstlisting}

\item Note that the default passwords are:
\begin{tabular}{l|l}
temppwd & for \texttt{debian} \\
root & for \texttt{root}
\end{tabular}

\end{itemize}




\subsection{Set LAN connection on BBB at AmP}

This is from:

\url{https://groups.google.com/forum/#!msg/beaglebone/AS2US9rtNd4/8y0mZ3LxAwAJ}

\begin{itemize}
\item You have to configure \texttt{eth0} like this:

\begin{tabular}{l l}
address & 134.28.136.51 (ask administrator for your personal IP)\\
netmask & 255.255.255.0 \\
dns-nameservers & 134.28.205.14 \\
gateway & 134.28.136.1
\end{tabular}

\item Plug in LAN cable.

\item Get the name of the LAN connection:
\begin{lstlisting}
su
root@beaglebone:/etc/network# connmanctl services
*Ac Wired                ethernet_689e19b50543_cable
\end{lstlisting}


\item Using the appropriate ethernet service, tell \texttt{connman} to setup a static IP address for this service.

Syntax: 
\begin{lstlisting}[breaklines=true]
connmanctl config <service> --ipv4 manual <ip_addr> <netmask> <gateway> --nameservers <dns_server>
\end{lstlisting}

In our case:
\begin{lstlisting}[breaklines=true]
connmanctl config ethernet_689e19b50543_cable --ipv4 manual 134.28.136.51 255.255.255.0 134.28.136.1 --nameservers 134.28.205.14
\end{lstlisting}


\item Reboot and you are done.


\item You can revert back to a DHCP configuration simply as follows:
\begin{lstlisting}
$ sudo connmanctl config ethernet_689e19b50543_cable --ipv4 dhcp
\end{lstlisting}

\end{itemize}


\subsection{Configure SSH Connection to BBB}
\begin{itemize}
\item Source: \url{https://askubuntu.com/questions/115151/how-to-set-up-passwordless-ssh-access-for-root-user}

\item If your Board crashed, and you were forced to reinstall the OS, there already exist a ssh-key.
This you have to remove first (this is for USB cable):
\begin{lstlisting}
bianca@bianca:~ ssh-keygen -f "/home/bianca/.ssh/known_hosts" -R 192.168.7.2
\end{lstlisting}

\item Generate a new key:
\begin{lstlisting}
bianca@bianca:~ ssh-keygen -f "/home/bianca/.ssh/key_bianca"
\end{lstlisting}
When you are prompted for a password, just hit the enter key and you will generate a key with no password.

\item Allow to log in as root with a password on the server, in aim to transfer the created key to it:
\begin{lstlisting}
root@beaglebone:# nano /etc/ssh/sshd_config
\end{lstlisting}
Make sure you allow root to log in with the following syntax
\begin{lstlisting}
PermitRootLogin yes
PasswordAuthentication yes
\end{lstlisting}
Restart the ssh-server: 
\begin{lstlisting}
root@beaglebone:# service ssh restart
\end{lstlisting}

\item Now you are able to transfer the key to the server:
\begin{lstlisting}
bianca@bianca:~ ssh-copy-id -i /home/bianca/.ssh/key_bianca root@192.168.7.2
\end{lstlisting}

\item Check if its work:
\begin{lstlisting}
bianca@bianca:~ ssh root@192.168.7.2
\end{lstlisting}

\item Now disable root login with password on server (for saftey):
\begin{lstlisting}
root@beaglebone:# nano /etc/ssh/sshd_config
\end{lstlisting}
And modify the Line:
\begin{lstlisting}
PermitRootLogin without-password
PasswordAuthentication yes
\end{lstlisting}
This will allow to login as root with valid key, but not with a password.
All other users can further login with a password.
Restart the ssh-server and you are done: 
\begin{lstlisting}
root@beaglebone:# service ssh restart
\end{lstlisting}
\end{itemize}



\subsection{Configure BBB Device Tree}

In order to enable \texttt{P9.28} as pwm pin, you have to load \texttt{cape-universala}. This you gonna do in \texttt{/boot/uEnv.txt}:

\begin{itemize}
\item source: \url{https://groups.google.com/forum/#!topic/beagleboard/EYSwmyxYjdM}

\item \texttt{/boot/uEnv.txt} should be looking something like this:

\begin{lstlisting}
root@beaglebone:# cat /boot/uEnv.txt | grep -v "#"

uname_r=4.4.54-ti-r93 
cmdline=coherent_pool=1M quiet cape_universal=enable
\end{lstlisting}

Edit it with:
\begin{lstlisting}
root@beaglebone:# nano /boot/uEnv.txt
\end{lstlisting}

Add the following lines, such that \texttt{/boot/uEnv.txt} looks like:
\begin{lstlisting}
root@beaglebone:# cat /boot/uEnv.txt | grep -v "#"

uname_r=4.4.54-ti-r93
dtb=am335x-boneblack-overlay.dtb
cmdline=coherent_pool=1M quiet cape_universal=enable
cape_enable=bone_capemgr.enable_partno=cape-universala
\end{lstlisting}
\item Reboot and you should be able to configure with:
\begin{lstlisting}
root@beaglebone:# config-pin P9_28 pwm
\end{lstlisting}
\end{itemize}


Note:
\begin{footnotesize}
\begin{itemize}
\item In \texttt{debian-elinux-version-9.1+} the \texttt{/boot/uEnv.txt} looks like:
\begin{lstlisting}
root@beaglebone:# cat /boot/uEnv.txt | grep -v "#"

uname_r=4.9.82-ti-r102
enable_uboot_overlays=1
enable_uboot_cape_universal=1
cmdline=coherent_pool=1M net.ifnames=0 quiet
\end{lstlisting}

If you see this, you may want to find a way to enable all the pins. I failed.

Robert C Nelson seems to be the only one, who has an idea whats going on...
\url{https://elinux.org/Beagleboard:BeagleBoneBlack_Debian#U-Boot_Overlays}

\end{itemize}
\end{footnotesize}


\subsection{Installing Software on BBB}
In order to run the \texttt{GeckoBot} software on the BBB install following packages:
\begin{itemize}
\item on BBB as \texttt{su}
\begin{lstlisting}
root@beaglebone:# apt-get update
root@beaglebone:# apt-get install ntpdate
root@beaglebone:# ntpdate pool.ntp.org
root@beaglebone:# apt-get install build-essential python-dev python-pip -y
root@beaglebone:# pip install --upgrade pip
root@beaglebone:# pip install Adafruit_BBIO
root@beaglebone:# pip install Adafruit_GPIO
root@beaglebone:# pip install termcolor
root@beaglebone:# pip install numpy

root@beaglebone:~# mkdir Git
root@beaglebone:~# cd Git
root@beaglebone:~/Git/# git clone https://github.com/larslevity/GeckoBot.git

\end{lstlisting}
\end{itemize}



\subsection{Running the Code}
To run the geckobot code:
\begin{itemize}
\item on BBB as \texttt{su}
\begin{lstlisting}
root@beaglebone:~# cd Git/GeckoBot/Code
root@beaglebone:~Git/GeckoBot/Code/# python server_hardware_controlled.py
\end{lstlisting}
\end{itemize}






\section{Pin Layout}

\begin{figure}[h!]
\begin{center}
\includegraphics[width=.95\textwidth]{Images/beaglebone-black-pinout.jpg}
\caption{Pin layout of BBB}
\end{center}
\end{figure}

The following pins were used, where Fx means foot x and y means leg or belly y:

\begin{table}[h!]
\begin{center}
\caption{Used pins (Outdated)}
\begin{tabular}{cccccccc}
\toprule
P8-7 	& P8-8 	& P8-9 		& P8-10 & P8-13 	& P8-19 	&		& 		\\
%\midrule
F1		& F2	& F3		& F4	& 2			& 1 		&		& 		\\
\midrule
P9-1 	& P9-5 	& P9-14 	& P9-16 & P9-19 	& P9-20		& P9-21	& P9-22 \\
%\midrule
VDD		& GND	& 6			& 5		& I2C-SCL	& I2C-SDA 	& 4		& 3 	\\
\bottomrule
\end{tabular}
\end{center}
\end{table}




\section{Auxilary}

\subsection{Formatting SD Card with debian}

\begin{itemize}
\item Source: \url{https://www.techwalla.com/articles/how-to-format-an-sd-card-in-debian-linux}

\item Determine location of SDCard (in the following called: \texttt{/dev/mmcblk0p2}) and directory where it is mounted (in the following called: \texttt{/media/SDCard}):
\begin{lstlisting}
su
df
\end{lstlisting}

\item Unmount, format, and remount:
\begin{lstlisting}
umount /dev/mmcblk0p2
mkdosfs /dev/mmcblk0p2 -F16
mount /dev/mmcblk0p2 /media/SDCard
\end{lstlisting}

\item For formatting SD with more than one partition, use:
\begin{lstlisting}
cfdisk /dev/mmcblk0
\end{lstlisting}
and follow the instructions.
\end{itemize}



\subsection{Set WiFi connection}

\begin{itemize}

\item Order WiFi Antenna \texttt{TP-LINK WLAN LITEN HI.G USB ADA. WN722N} from somewhere.

\item Complete this tuturial ...

\end{itemize}



\subsection{Setup for analog inputs}

\begin{itemize}

\item \url{https://groups.google.com/forum/#!topic/beagleboard/Lk3vWNIExiQ}

\item Insert in command line on BBB:

\begin{lstlisting}[language=bash]
su apt-get install bb-cape-overlays

cd /opt/source/bb.org-overlays

./dtc-overlay.sh

./install.sh

sudo sh -c "echo 'BB-ADC' > /sys/devices/platform/bone_capemgr/slots"
\end{lstlisting}


\item Reboot.

\item For readout the ADC input Pins from python: \url{https://learn.adafruit.com/setting-up-io-python-library-on-beaglebone-black/adc}


\end{itemize}




\end{document}
